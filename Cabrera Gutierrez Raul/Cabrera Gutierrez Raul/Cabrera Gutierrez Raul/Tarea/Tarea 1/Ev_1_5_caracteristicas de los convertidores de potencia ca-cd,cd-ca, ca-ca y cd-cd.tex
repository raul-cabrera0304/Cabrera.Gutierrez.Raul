\documentclass[11pt]{article}
\usepackage[utf8]{inputenc}
%Gummi|065|=)
\title{\textbf{ Caracterìsticas de los convertidores de potencia CA-CD, CD-CA, CA-CA-CA Y CD-CD}}
\author{Cabrera Gutièrrez Raùl\\
               \\
		Sistemas elèctronicos de interfaz}
		

\date{}
\usepackage{graphicx}
\begin{document}
\begin{figure}[htp]
\centering
\includegraphics[scale=1.50]{/home/raulcb/Downloads/th.jpeg}
\caption{logo}
\label{logo}
\end{figure}
\maketitle


\section{intoduccion}
La conversion de potencia es el proceso de convertìr una forma de energìa en otra, esto puede incluir electromagnètico o electroquìmicos.
Un convertidor elèctronico es en realidad un dispositivo que se comporta como un interruptor y que esta construido con semiconductores, ya sean diodos, transistores de potencìa, tiristores, GTO, IGBT, BJT, transistores MOSFET, tiristores MCT. 
\section{Diodo rectificador}
Es un componente elèctronico cosntruido a base de una misiòn de semiconductores N-P. Dispone de 2 terminales, el ànodo (A) quecorresponde al semiconductor P y, el càtodo (K) que corresponde con el semiconductor N. 
Hay muchas clases de diosos, pero los que nos interesan a nosotros son los de alta potencia, es decir, los diòdos rectificadores capaces de soportar corrientes superiores de a 1A. Existen diòdos rectificadores capaces de soportra 5000A y tensiones de pico inversas de 5KV.  
\section{Tirirstores}
Tambien conocidos como SCR o semiconductor controlled rectifiers. Los tiristores son componentes deelectronicos de compuestos por 3 uniones de semiconductores P-N-P-N. Dispone de 3 terminales, un ànodo (A), un càtodo (K) y una puerta (G).
\section{Tipos de convertidores}
\textbf{Convertidores de C.A a C.C:} Transforma una señal de c.a de entrada de una señal c.c de salida. Tambien son llamados rectificadores. La corriente continùa se puede variar por medio de un tiristor. Este tipo de convertidor se sule utilizar para variar la velocidad de los motores c.c

\textbf{Convertidores de C.C a C.C:} Se transforma una señal de c.c de entrada de otra señal c.c de salida. Tambièn llamado inversores u onduladores. Modificando la frecuencia de la onda c.a se controla la velocidad de los motores c.a. En la actualidad se utilizan para regular la velocidad de trenes con motores de c.a asìncronos que circulan por vìas de corriente continua.


\textbf{Convertidores de C.C a C.A.:} Transforma una señal dec.c de entrada en otra señal c.c de salida de mayor o menor amplitud, segùn la necesidad. Tambièn son conocidos con los nombres de choppers o recortadores. Son muy utilizados para regular la velocidad de los motores elèctricos c.c de tranvìas, trenes, etc.

\textbf{\begin{center}Convertidores de C.A a C.A.: Existen 3 clases de convertidores c.a ac.a.:\end{center}}

\textbf{-Convertidores C.A/C.C./C.A.:} Como su nombre lo indica, hay una transformaciòn intermedia de c.a. en c.c. para luego disponer de una salida de señal de c.a.. En la primera etapa de transformaciòn de c.a.a c.c. se utilizan tiristores o rectificadores. En la segunda etapa, la transformaciòn de c.c. a c.a. se utilizan inversores. Se utilizan para controlar la velocidad de motores elèctricos de c.a. sìncronos y asìncronos. Son utilizados para arrancar y regular la velocidad en trenes de alta velocidad como el AVE.

\textbf{-Convertidor DC-DC}

Se llama convertidor DC-DC a un dispositivo que transforma corriente continua de uan tension a otyra. Suelen ser reguladores de conmutaciòn, dando a su salida uan tensiòn regulada y, la mayorìa de las veces con limitaciones de corriente. Se detiene a utilizar la frecuencias de conmutaciòn cada vez màs elevadas porque permiten reducir la capacidad de los condensadores, con el consiguiente beneficio de volumne, peso y precio.
\end{document}
