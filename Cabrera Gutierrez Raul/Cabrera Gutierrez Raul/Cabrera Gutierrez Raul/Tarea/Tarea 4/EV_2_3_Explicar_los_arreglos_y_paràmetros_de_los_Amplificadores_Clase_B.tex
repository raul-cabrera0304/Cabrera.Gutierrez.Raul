\documentclass[11pt]{article}
\usepackage[utf8]{inputenc}
%Gummi|065|=)
\title{\textbf{Explicar los arreglos y paràmetros de los amplificadores clase B }}
\author{Cabrera Gutièrrez Raùl
\\
Sistemas electrònicos de interfaz}

\date{08-Oct-2019}
\usepackage{graphicx}
\begin{document}

\maketitle

\section{Amplificadores clase B}
Los amplificadores clase B se caracterizan por tener intensidad casi nula a travès de sus transistores cuando no hay señal en la entrada del circuito, por lo 	que en reposo el consumo es casi nulo.
\\

Uno de los principales inconvenientes de los amplificadores en clase A es que, en reposo, estàn consumiendo corriente por lo que el rendimiento de conversiòn se hace bastante bajo. Para mejorar este rendimiento, y por tanto aprovechar al màximo la potencia entregada por la fuente de alimentaciòn, los amplificadores se suelen construir en clase B.


Por norma general, los amplificadores que se van a hacer trabajar en clase B, se montan con transistores que trabajen en contrafase (push-pull) con el fin de minimizar los armònicos que se pueden generar en este tipo de montajes, estos amplificadores adoptan uan serie de montajes determinados.

\begin{figure}[htp]
\centering
\includegraphics[scale=0.40]{/home/raulcb/Downloads/generalidades.jpg}
\caption{.}
\label{.}
\end{figure}

\section{Caracterìsticas}
se le denomina amplificador clase B, cuando el voltaje de polarizaciòn y la màxima amplitud de la señal entrante poseen valores que hacen que la corriente de salida circule durante el semiciclo de la señal de entrada.


Las caracterìstica principal de este tipo de amplificadores es el alto factor de amplificaciòn.


Los amplificadores de clase B usan dos o màs transitores polarizados de tal forma que cada transistor solo conduce durante un medio ciclo de laonda de entrada.


\section{Ventajas}
Elamplificador de clase B tiene la gran ventaja sobre sus primos de amplificador de clase A en que ninguna corriente fluye a tràves de los transistores cuando estàn en estado de reposo(es decir, sin señal de entrada), por lo tanto no se disipa potencia en los transistores de salida o transformador cuando no hay señal presente a diferencia de las etapas de amplificador de clase A que requieren un sesgo de base significativo, disipando asì gran cantidad de calor, incluso sin presencia de señal de entrada.

\begin{figure}[htp]
\centering
\includegraphics[scale=0.50]{/home/raulcb/Downloads/Amplificador de clase B.jpg}
\caption{.}
\label{.}
\end{figure}

\end{document}
