\documentclass[11pt]{article}
%Gummi|065|=)
\usepackage[utf8]{inputenc}
\title{\textbf{Explicar los parametros de los amplificadores clase A}}
\author{Raùl Cabrera Gutièrrez
    \\Sistemas elèctricos de interfaz
    \\Mecatrònica 4ª}
\date{1 de Octubre del 2019}
\usepackage{graphicx}
\begin{document}

\maketitle

\section{Amplificadores clase A}
Son aquellos amplificadores cuyas etapas de potencia consumen corrientes altas y continuas de su fuente de alimentaciòn, independientemente de si existe.
Un amplificador de potencia fuciona en calse A cuando la tensiòn de polarizaciòn y la amplitud màxima de la señal circule durante todo el perìodo de la señal de entrada.

\begin{figure}[htp]
\centering
\includegraphics[scale=0.50]{/home/raulcb/Downloads/AmplificadorclaseA.jpg}
\caption{.}
\label{.}
\end{figure}
\section{Caracterìsticas}
Esta amplificaciòn presenta el inconveniente de generar una fuente y constante emisiòn de calor. No obstante, los transistores de salida estan en una temperatura fija y sin alteraciones.
Los amplificadores de clase A a menudo consisten un transistor de salida conectado al positivo de la fuente de alimentaciòn y un transistor de corriente constante conectado de la salida al negativo de la fuente de alimentaciòn.

En los amplificadores de clase A no hay nunca corriente de reja (base) por lo que es indiferente decir que el amplificador es de clase A1 o de clase A.

\begin{figure}[htp]
\centering
\includegraphics[scale=0.50]{/home/raulcb/Downloads/Amplificador clase A.jpg}
\caption{.}
\label{.}
\end{figure}


\section{Ventaja}
La clase A se refiere a una etapa de salida con uuna corriente de polarizaciòn mayor que la màxima corriente de salida que dan, de tal forma que los transistores de salida siempre estàn consumindo corriente. La gran ventaja de la clase A es que es casi lineal y en consecuencia la distorciòn es menor.

\section{Desventaja}
La gran desventaja de la clase A es que es poco eficiente, se requiere un amplificador de clase A muy grande para dar 50W y ese apmlificador usa mucha corriente y se pone a muy alta temperatura.
\end{document}
