\documentclass[11pt]{article}
\usepackage[utf8]{inputenc}
%Gummi|065|=)
\title{\textbf{Giro de un moro de corriente directa}}
\author{Cabrera Gutièrrez Raùl}
\date{15-0ct-2019}
\usepackage{graphicx}
\begin{document}

\maketitle

\section{Motor DC}
 Los motores de Corriente Directa o motor DC(correspondiente a las iniciales en inglés “direct current”) es también conocidos como motor de Corriente Continua o motor CC, son  muy utilizados en diseños de ingeniería debido a las características torque-velocidad que poseen con diferentes configuraciones eléctricas o mecánicas.
 \\
 Una gran ventaja de los motores de CD se debe a que es posible controlarlos con suavidad y en la mayoría de los casos son reversibles, responden rápidamente gracias a que cuentan con una gran razón de torque a la inercia del rotor. 
 \\
 Otra ventaja es la implementación del frenado dinámico, donde la energía generada por el motor se alimenta a un resistor disipador, y el frenado regenerativo donde la energía generada por el motor retroalimenta al suministro de potencia CD, esto es muy utilizado en aplicaciones donde se deseen frenados rápidos y de gran eficiencia.
 \begin{figure}[htp]
\centering
\includegraphics[scale=.50]{/home/raulcb/Downloads/images.jpeg}
\caption{.}
\label{.}
\end{figure}

\section{Control del sentido de giro del motor}
El control del sentido de giro se realix¡za por medio de un circuito conocido como "puete H".
Pra el control de velocidad de motores de corriente directa se utiliza en general la tècnica conocida como PWM o Pulse Wide Modulation. El mètodo consiste en el switcheo ràpido (alrededor de 20khz) de la fuente de alimentaciòn del motor, proporcionado a èste una potencia promedio controlada mediante el ancho de los pulsos.
\\
Bàsicamente es un circuito electrònico que permite invertir el sentido de la corriente directa en el motor, cambiando de esta forma su direcciòn de giro.
\begin{figure}[htp]
\centering
\includegraphics[scale=.50]{/home/raulcb/Downloads/H-BRIDGE.jpg}
\caption{.}
\label{.}
\end{figure}


\end{document}
