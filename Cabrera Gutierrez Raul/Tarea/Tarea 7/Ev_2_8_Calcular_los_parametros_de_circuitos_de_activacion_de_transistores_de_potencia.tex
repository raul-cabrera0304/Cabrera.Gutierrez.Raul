\documentclass[11pt]{article}
\usepackage [utf8]{inputenc}
\usepackage{fourier}
%Gummi|065|=)
\title{\textbf{Calcular los parametros de circuitos de activación de transistores de potencia}}
\author{Cabrera Gutièrrez Raùl
\\ Mecatrònica 4ª
\\
Sistemas electrònicos de interfaz}
\date{}
\begin{document}

\maketitle

\section{Transistor de potencia}

 El transistor de potencia es un dispòsitivo electrònico semiconductor utilizado para entregar una señal de salida en respuesta a una señal de entrada. Cumple funciones de amplificador, oscilador, conmutador o rectificador.
 El tèrmino "transistor" es la contracciòn en inglès de transfer resistor.
\section{Càlculos}
Par este proceso de los càlculos es necesario conocer cada uno de los datos "datasheet" para saber como funciona y de que forma trabaja.

Datos del TIP41:\\
\begin{itemize}
\item Ic: (max=6A, min= 25mA.)
\item Ib: (max=3A, min=15mA)
\item PTOT=65W\\
\item HFE: 15 a 75\\
\item VCE: 4v\\
\item VEBO: 5V\\
\item VCEO: 100V\\
\item VCBO: 100V\\
\end{itemize}


Ya que se han obtenido los datos, se comienzan los càlculos.
\\
Se comienza encontrando los valores de RB, tomando en cuenta las base del transistor  esta conectado de froma directa al transistor de corriente.
\\
Utilizamos lo siguiente:

$$ R_{B}=\frac{V_{cc}-V_{D12}}{ \frac{Ic}{HFE} } $$

Para lo cual tenemos que sustituir los valores ya puestos.
\\
$$ R_{B}=\frac{9v-1.4v}{ \frac{0.52}{75}}= 1,134.33\Omega $$
\\

Lo dejamos en 1200ohms

Con el valor de la primer resistencia proceguiremos en sacar el valor de la segunda.


$$ RC=V_CB+V_CC-I_C=V/I $$

Queda un poco màs claro como es el comportamiento de la entrada y esto ayuda a que no se sature lo que se va a conectar en 127v.

$$ R_{C}=V_{CB}+V_{CC}-I_{C}=7.6v/0.52mA= 146.15\Omega $$

Quedando en 150ohms

Ya conlos dos valores obtenidos de las resistencias ya que tiene que funcionar el transistor como switch, en esta parte tenemos claro que con cuanto trabaja la base y la del emisor.
\\
seguimos con el Ic 

$$ I_{C}=\frac{V_{CC}-V_{CE}}{R_{B}} $$

Para esta parte solo trabajamos en la reaccion de la transmicion de la corriente.

$$ I_{C}=\frac{9v-4v}{146.15\Omega}= 34.2mA $$


Ahora tenemos que sacar la corriente de la baseosea, la corriente que reciba el IB tiene que ser sufuciente para que est eactue comop swuich.

$$ I_{B}=\frac{V_{CBO}-V_{D12}-V_{EBO}}{R_{B}} $$


Tenemos que ver la corriente que va a recibir la base, para la activacion de emisor, asi que:


$$ I_{B}=\frac{100V-1.4V-5V}{1,134.33\Omega}= 0.082mA $$

Esta siendo la corriente que recibe para que encienda el emisor


Ahora la corriente con la que encendera el emisor en lo mas minimo.
$$ I_{C}=\frac{V_{CC}-V_{EBO}-V_{D12}}{R_{C}} $$

Esto nos ayuda a tener mayor control ya que se regula el voltaje antes de llegar a la base del transistor.


$$ I_{C}=\frac{9V-5V-1.4V}{1,134.33\Omega}= 2.3mA $$

Para otra parte se debe ver la potencia con la que se estara trabajando.

$$ P=R_{C}*I_{B}^{2} $$


Es solo la ley de Ohm, y queda de esta manera


$$ P= 1,134.33\Omega*82mA^{2}= 7.63\omega $$






\end{document}
